\section{Profiling and Debugging}

Efficient debugging of parallel workloads often requires deep visibility into the execution state. To support this, the system implements an \textbf{Enhanced Trace} mechanism that streams cycle-accurate execution data in JSON format over the high-speed UART.

\subsection{Enhanced Trace Format}

The trace logger captures a comprehensive snapshot of the machine state for every instruction executed. As shown in Listing \ref{lst:json_trace}, the format is structured to facilitate offline analysis and visualization.

\begin{lstlisting}[caption={Example JSON Trace Record}, label={lst:json_trace}, basicstyle=\ttfamily\scriptsize]
{
  "cycle": 152,
  "pc": 12,
  "instruction": "0x11040203",
  "asm": "IADD R4, R2, R3",
  "exec_time_us": 125,
  "hw_ctx": {
    "sm_id": 0,
    "warp_id": 0,
    "lane_id": 0,
    "active_mask": "0xFF"
  },
  "perf": {
    "latency": 1,
    "stall_cycles": 0,
    "stall_reason": "NONE",
    "pipe_stage": "WRITEBACK"
  },
  "lanes": [
    {
      "lane_id": 0,
      "R": [5, 3, 8, ... ]  // Full Register Dump
    }
  ]
}
\end{lstlisting}

\subsection{Key Features}

\subsubsection{Hardware Context}
The \texttt{hw\_ctx} object encapsulates the SIMT topology, providing the specific Streaming Multiprocessor (SM), Warp, and Lane ID for the instruction. The \texttt{active\_mask} reveals thread divergence handling.

\subsubsection{Performance Metrics}
The \texttt{perf} object exposes the micro-architectural behavior:
\begin{itemize}
    \item \textbf{\texttt{exec\_time\_us}}: Real-time execution duration measured via high-resolution timers.
    \item \textbf{\texttt{stall\_cycles}}: Pipeline bubbles caused by data dependencies or memory latency.
    \item \textbf{\texttt{pipe\_stage}}: The current pipeline stage (e.g., DECODE, EXECUTE, WRITEBACK).
    \item \textbf{\texttt{sync\_barrier}}: Automatic detection of \texttt{BAR.SYNC} operations.
\end{itemize}

\subsubsection{Register Inspection}
Unlike traditional debuggers that inspect state at breakpoints, the Enhanced Trace dumps the complete register file (\texttt{R0-R23}) for all active lanes, enabling a "time-travel" debugging experience where variable values can be tracked historically.

\subsection{Trace Usage}
Tracing is enabled via the \texttt{trace:stream} command. Due to the high bandwidth requirements, it is strongly recommended to use Turbo Mode (460,800 baud) when this feature is active. The host-side Python client parses this stream to generate timeline visualizations similar to Nsight Compute.

\subsection{Host-Side Visualization}
The Python client parses the JSON stream to generate interactive timelines.

\subsubsection{Warp Divergence Analysis}
The visualizer reconstructs the execution path of each warp, highlighting divergent branches (where \texttt{active\_mask != 0xFF}) in red. This allows developers to identify inefficient kernels that suffer from serialization.

\subsubsection{Pipeline Stall Visualization}
By tracking \texttt{stall\_cycles}, the tool identifies memory bottlenecks. A heatmap view correlates high latency instructions with specific VRAM banks, aiding in memory layout optimization.
