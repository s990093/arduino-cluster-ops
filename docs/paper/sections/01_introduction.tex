\section{Introduction}
As Deep Learning models like Transformers move to the edge, the need for efficient matrix operations on embedded devices grows. While high-end edge GPUs (e.g., NVIDIA Jetson) exist, there is a significant gap in enabling GPU-like programming models on ubiquitous, low-cost microcontrollers like the ESP32.

Traditional Microcontroller Units (MCUs) operate on a SISD (Single Instruction Single Data) model. Emulating a GPU requires a fundamental architectural shift to SIMT (Single Instruction Multiple Threads). This paper proposes a novel approach to emulate a Streaming Multiprocessor (SM) using the asymmetric dual-core architecture of the ESP32.

Our contributions are as follows:
\begin{enumerate}
    \item \textbf{Dual-Core Split Architecture}: We decouple control logic (Core 0) from execution logic (Core 1) to mimic the GPU Front-End/Back-End split, utilizing FreeRTOS queues for synchronization.
    \item \textbf{Micro-CUDA ISA v1.5}: We introduce a custom 32-bit instruction set supporting predicated execution, warp synchronization, and lane-based addressing, specifically designed for MCU constraints.
    \item \textbf{Virtual SIMD Engine}: A runtime environment on Core 1 that manages 8 virtual "lanes" with isolated register contexts, enabling true data parallelism.
\end{enumerate}
